\documentclass{article}

\usepackage{fancyhdr}
\usepackage{extramarks}
\usepackage{boondox-cal}
\usepackage{amsmath}
\usepackage{amsthm}
\usepackage{amsfonts}
\usepackage{tikz}
\usepackage{svg}
\usepackage[plain]{algorithm}
\usepackage{algpseudocode}

\usetikzlibrary{automata,positioning, arrows.meta}

%
% Basic Document Settings
%

\topmargin=-0.45in
\evensidemargin=0in
\oddsidemargin=0in
\textwidth=6.5in
\textheight=9.0in
\headsep=0.25in

\linespread{1.1}

\pagestyle{fancy}
\lhead{\hmwkAuthorName}
\chead{\hmwkClass\ (\hmwkClassInstructor): \hmwkTitle}
\rhead{\firstxmark}
\lfoot{\lastxmark}
\cfoot{\thepage}

\renewcommand\headrulewidth{0.4pt}
\renewcommand\footrulewidth{0.4pt}

\setlength\parindent{0pt}

%
% Create Problem Sections
%

\newcommand{\enterProblemHeader}[1]{
    \nobreak\extramarks{}{Problem \arabic{#1} continued on next page\ldots}\nobreak{}
    \nobreak\extramarks{Problem \arabic{#1} (continued)}{Problem \arabic{#1} continued on next page\ldots}\nobreak{}
}

\newcommand{\exitProblemHeader}[1]{
    \nobreak\extramarks{Problem \arabic{#1} (continued)}{Problem \arabic{#1} continued on next page\ldots}\nobreak{}
    \stepcounter{#1}
    \nobreak\extramarks{Problem \arabic{#1}}{}\nobreak{}
}

\setcounter{secnumdepth}{0}
\newcounter{partCounter}
\newcounter{homeworkProblemCounter}
\setcounter{homeworkProblemCounter}{1}
\nobreak\extramarks{Problem \arabic{homeworkProblemCounter}}{}\nobreak{}

%
% Homework Problem Environment
%
% This environment takes an optional argument. When given, it will adjust the
% problem counter. This is useful for when the problems given for your
% assignment aren't sequential. See the last 3 problems of this template for an
% example.
%
\newenvironment{homeworkProblem}[1][-1]{
    \ifnum#1>0
        \setcounter{homeworkProblemCounter}{#1}
    \fi
    \section{Problem \arabic{homeworkProblemCounter}}
    \setcounter{partCounter}{1}
    \enterProblemHeader{homeworkProblemCounter}
}{
    \exitProblemHeader{homeworkProblemCounter}
}

%
% Homework Details
%   - Title
%   - Due date
%   - Class
%   - Section/Time
%   - Instructor
%   - Author
%

\newcommand{\hmwkTitle}{Problem Set\ \#1}
\newcommand{\hmwkDueDate}{February 28, 2024}
\newcommand{\hmwkClass}{CS229}
\newcommand{\hmwkClassInstructor}{Andrew Ng}
\newcommand{\hmwkAuthorName}{\textbf{0130}}

%
% Title Page
%

\title{
    \vspace{2in}
    \textmd{\textbf{\hmwkClass:\ \hmwkTitle}}\\
    \vspace{0.1in}\large{\textit{\hmwkClassInstructor}}
    \vspace{3in}
}

\author{\hmwkAuthorName}
\date{}

\renewcommand{\part}[1]{\textbf{\large Part \Alph{partCounter}}\stepcounter{partCounter}\\}

%
% Various Helper Commands
%

% Useful for algorithms
\newcommand{\alg}[1]{\textsc{\bfseries \footnotesize #1}}

% For derivatives
\newcommand{\deriv}[1]{\frac{\mathrm{d}}{\mathrm{d}x} (#1)}

% For partial derivatives
\newcommand{\pderiv}[2]{\frac{\partial}{\partial #1} (#2)}

% Integral dx
\newcommand{\dx}{\mathrm{d}x}

% Alias for the Solution section header
\newcommand{\solution}{\textbf{\large Solution}}

% Probability commands: Expectation, Variance, Covariance, Bias
\newcommand{\E}{\mathrm{E}}
\newcommand{\Var}{\mathrm{Var}}
\newcommand{\Cov}{\mathrm{Cov}}
\newcommand{\Bias}{\mathrm{Bias}}

% empty underline
\newcommand{\emptyunderline}{\underline{\ \ \ \ \ \ }}

\begin{document}

\maketitle

\pagebreak

\begin{homeworkProblem}
    \begin{enumerate}
        \item \[
                \begin{aligned}
                \frac{\partial}{\partial \theta_j} J(\theta) &= \frac{\partial}{\partial \theta_j} -\frac{1}{m} \sum_{i=1}^m y^{(i)} \log (h_{\theta}(x^{(i)})) + (1 -
                y^{(i)} \log (1 - h_{\theta}(x^{(i)}))) \\
                &= - \frac{1}{m} \sum_{i=1}^m [ y^{(i)} \frac{1}{h_{\theta}(x^{(i)})}  - (1 - y^{(i)}) \frac{1}{1-h_{\theta}(x^{(i)})} ] \frac{\partial}{\partial \theta_j} h_{\theta}(x^{(i)})
                \end{aligned}
            \]
            Then we calculate:
            \[ 
                \frac{\partial}{\partial\theta_j} h_{\theta}(x^{(i)}) = \frac{\partial}{\partial \theta_j} g(\theta^T x^{(i)}) = g(\theta^T x^{(i)})(1-g(\theta^T x^{(i)})) x_j^{(i)} = h_{\theta}(x^{(i)})(1 - h_{\theta}(x^{(i)})) x_j^{(i)}
            \]
            And we can further simplify the above equation:
            \[
                \begin{aligned}
                \frac{\partial}{\partial \theta_j} J(\theta) &= -\frac{1}{m} \sum_{i=1}^m [(y^{(i)} - h_\theta(x^{(i)})) - (1-y^{(i)})h_{\theta}(x^{(i)})] x_{j}^{(i)} \\
                &= -\frac{1}{m} \sum_{i=1}^m (y^{(i)} - h_{\theta}(x^{(i)})) x_j^{(i)} \\
                \end{aligned}
            \]

            Then we can get second-order derivative:

            \[
                \begin{aligned}
                    H_{ij} = \frac{\partial^2}{\partial \theta_i \partial \theta_j} J(\theta)  &= \frac{1}{m} \sum_{k=1}^m x_j \frac{\partial}{\partial \theta_i} h_{\theta}(x^{(k)}) \\
                    &= \frac{1}{m} \sum_{k=1}^m x_{i}^{(k)} x_{j}^{(k)} h_{\theta}(x^{(k)})(1-h_{\theta}(x^{(k)}))\\
                \end{aligned}
            \]
            for each vector z, consider the quadratic form of Hessian matrix:

            \[
                \begin{aligned}
                    z^T H z &= \sum_{i=1}^n \sum_{j=1}^n z_i H_{ij} z_j = \frac{1}{m} \sum_{i=1}^n \sum_{j=1}^n \sum_{k=1}^m z_i x_i^{(k)} z_j x_j^{(k)}h_\theta(x^{(k)})(1-h_{\theta}(x^{(k)})) \\
                    &= \frac{1}{m} \sum_{k=1}^m (\sum_{i=1}^n \sum_{j=1}^n z_i x_i^{(k)} z_j x_{j}^{(k)}) h_{\theta}(x^{(k)}) (1-h_\theta(x^{(k)})) \\
                    &= \frac{1}{m}\sum_{k=1}^m {(x^{(k)T}z)}^2 h_{\theta}(x^{(k)}) (1 - h_\theta(x^{(k)})) \geq 0 \Leftrightarrow H \succeq 0
                \end{aligned}
            \]
        \item Codes are shown in src director.
        \item 
            \[
                1
            \]
    \end{enumerate}
\end{homeworkProblem}

\pagebreak

\end{document}